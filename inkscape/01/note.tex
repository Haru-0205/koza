\documentclass{ltjsarticle}
\begin{document}

\title{名刺を作ってみよう!}
\author{谷口陽音}
\date{2023年7月9日}
\maketitle

\begin{abstract}
  このドキュメントは、第1回 Inkscapeを用いたデザインワークショップのテキストとして作成されたものである。\\
  今回においては、前半はInkscape等の概要について解説した後、後半では実際に制作を行う。\\
  今回の資料類については、GitHubにてすべて公開している。詳細はワークショップ中に説明する。各自適切に活用すること。\\

  なお、このドキュメントにおける一切の権利については、著者である谷口に帰属する。\\
  よって、著者の許可なくこのドキュメントを無断転載或いは再配布等を行うことは固く禁じる。
\end{abstract}

\section{Inkscapeについて}
\subsection{ラスターとベクター}
我々が書いているデジタルイラスト・撮っている写真・作っているレポート\dots \\
そのすべてに拡張子がついている。そして、ファイルの役割は主にその拡張子によって決まる。\\
以下に代表的な拡張子とその役割を挙げる。\\
\fbox{.txt}\dots テキストファイル
\fbox{.jpeg}\fbox{.png}\fbox{.svg}\fbox{.bmp}\dots 画像ファイル
\fbox{.docs}\fbox{.xlsx}\fbox{.pptx}\dots Microsoft Officeファイル
\subsubsection{ラスター画像とは}

\end{document}