\documentclass{ltjsarticle}
\begin{document}

\title{Webについて}
\author{谷口  陽音}
\maketitle

\section{Webページの仕組み}
\subsection{Webページの構成}
我々が見ているWebページは、主に3種類程度のファイルから構成されている。
\subsubsection{HTML}
HTMLは、Hyper Text Markup Languageの略であり、主にWebページの基本的なパーツを記述する。\\
HTMLの最新版はHTML5\footnote[1]
{ここでは便宜上HTML5としたが、正確には「HTML Living Standard」である。しかし、現在において「HTML5」は広く浸透しており、かつこうなった背景にはかなり複雑な歴史があるため、ここでは割愛する。しかし、経緯自体はかなり興味深いものであるため、気になる人はWeb検索をかけてみることを推奨する。}\
であり、これはオープンソース\footnote[2] {オープンソースとは、GitHubなどでソースコードを公開しながら開発された製品や、その開発のことを指す。なお、このような言語は殆どの場合オープンソースである。詳しくはJavaScriptの詳細のところでふれる。}である。\\
HTMLは「タグ」というものを用いて記述する、特徴的な言語である。

\subsubsection{CSS}
CSSは、Cascading Style Sheetsの略であり、主にHTMLで記述された要素の見た目やWebページのレイアウトを制御する。\\
CSSの最新版は2011年から策定が進められているCSS3\footnote[3]{まだ正式に勧告されていない(執筆当時)。}であり、これもまたオープンソースである。\\
なお、CSSはMarkdownのレイアウト調整などにも用いることができる。

\subsubsection{JavaScript}
JavaScriptは、主にWebブラウザ上で動作するプログラミング言語\footnote[4]{HTMLはマークアップ言語、CSSはスタイルシートであり、いずれもプログラミング言語ではない}  であり、Webプログラミング界のデファクト・スタンダード\footnote[5]{「業界標準」の意。} である。\\
JavaScriptも例によってオープンソースである。\\
言語仕様の面ではJavaScriptとJavaは全くもって関連性がない\footnote[6]{これもHTML同様少々複雑なため説明を割愛する。なお、これについても是非ともWebサイトで調べてみてほしい。なお、現在商標権を持っているのはOracleであるが、JavaScriptのネイティブのブラウザはFirefoxであるため、混同しないよう十分に注意する必要がある。}ことに注意が必要である。

\subsection{フロントエンドとバックエンド}
Webページには、ユーザーが直接見れる部分と直接は見れない部分が存在する。\\
前者を\underline{フロントエンド}、後者を\underline{バックエンド}という。\\
フロントエンドは\S 1.1で触れた内容である。
バックエンドは次の\S 3.3 で触れる。

\subsection{バックエンド}
\S 1.2 で述べた通り、Webページの利用者から見えない部分をバックエンドと呼ぶ。\\
バックエンドでは次のようなことを行う。

\subsubsection{データベース}
まず、バックエンド側はフロントエンド側から受け取った情報を基にデータベースを操作することが1つの大きな仕事である。\\
ここで使われるのが\underline{SQL}である。\\
SQLは、「SQL文」というものを使ってデータベースにデータを追加・削除・編集したり、必要な情報を必要なぶんだけ取得したりする。\\
ここで、前者の「データ」を「\underline{レコード}」といい、後者のときに取得したデータを「\underline{クエリ}」、データを取得することを「\underline{抽出する}」という。

\subsubsection{WebAPI}
次に、バックエンド側の処理として、WebAPIを作成し、Web上のデータの処理を行うことが挙げられる。\\
WebAPIとは、Webページ関係のプログラミングを行うためのインターフェース\footnote[7]{入出力用の端子のようなもの。Webページをプロジェクターとし、PCをプログラムとすると、PCをプロジェクターにケーブルでつなげることで、どこに何を表示するのか、プロジェクターに命令を出すことができる。}である。\

\subsubsection{WebAPIの必要性}
ところで、先程WebAPIはプログラミングAPIと言ったが、Webページのフロントエンド側にもJavaScriptが実装されているではないか、と思う人もいるだろう。\\
両者の違いは非常に簡単である。次の\S 1.3.4で詳細は述べるが、今簡単に説明しておくとプログラムが実行される場所が違い、それぞれにメリットがある、ということになる。\\
したがって、現在のWebにおいては、WebAPIは必要不可欠な技術である。

\subsubsection{Ruby}
Rubyは、日本人のまつもとゆきひろ氏が開発したプログラミング言語である。\\
バックエンド開発によく使われるが、「Ruby on Rails」というフレームワーク\footnote[8]{「枠組み」の意。プログラミング用のテンプレートのようなもの。}を用いるとフロントエンドにも使える。\\
どちらかというと「楽しく書ける」言語であり、日本語ドキュメントも充実している。

\subsubsection{Node.JS}
Node.JSはJavaScriptをサーバー上\footnote[9]{サーバー上で直接動かす環境のこと。普通の実行環境はブラウザ上の環境なのでサーバーでは直接動かせない。}で動かせるようにしたもの。\\
JavaScriptで簡潔にサーバーサイドのプログラムを書くことができる。\\
npmというパッケージマネージャーがあり、様々なライブラリがそこからインストールできる。

\subsubsection{Java}
定番中の定番。\\
上の2つに比べて記述が難しく、プログラムが長くなりやすいので最近はあまり使われない。\\
「Java VM」上で動作するのでOSに関係なく同じように動作する。\\
数年前まではAndroid開発の公式言語としてGoogleが推奨していた\footnote[10]{Javaはライセンス関係が複雑で、最近商用ライセンスが有料化されたため、「Java離れ」はより加速することが予測される。なお、現在GoogleはJavaの上位互換を目指して開発されている「Kotlin」を推奨している。なお、「Kotlin」のライセンス元はチェコのJetBrains社である。}

\subsubsection{Python}
なんだかんだでPythonも動く。\\
Pythonは科学計算用途に強く、そのようなことをやるときはPythonを用いることがある。\\
最近はRubyに押され気味である。

\end{document}